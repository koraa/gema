% $Id$

\section{Customized command-line processing}

Part of the intent of {\bf gema} is that it can be used as a means of
implementing more specialized tools.  A utility program is defined by
the command line arguments that it uses as well as by how it processes
its input files.  Therefore, {\bf gema} provides a way to customize the
handling of command line arguments.

The main program of {\bf gema} just does some initialization, and then
processes the command line arguments by translating them with a set of
built-in patterns.  These rules that define the command line arguments
are defined in a domain named ``{\tt ARGV}''.  The user is free to add
additional rules to this domain, thereby implementing new command line
options, or even to undefine existing rules.
In the input stream that is translated by the {\tt ARGV}
domain, the command line arguments are separated by newline
characters.\footnote{The newline was chosen for
convenience, but it would more exactly emulate the C argument
semantics if the NUL character was used as the separator, and perhaps
that ought to be done in the future.}
The actions for the {\tt ARGV} rules are expected to do all their work
with side-effects and to not return any value.  Any value that is
returned by the translation (except for the delimiting newlines) will be
reported by the main program as undefined arguments.

The complete set of built-in {\tt ARGV} rules can be seen by looking at
the source file ``{\tt gema.c}'' in the variable {\tt argv\_rules}.
Here are a few representative examples:
\begin{verbatim}
  ARGV:\N-idchars\n*\n=@set-parm{idchars;$1}
  ARGV:\N-literal\n*\n=@set-syntax{L;$1}
  ARGV:\N-p\n*\n=@define{*}
  ARGV:\N\L*\=*\n=@define{$0}
  ARGV:\N-odir\n*\n=@set{.ODIR;*}
  ARGV:\N-<L1>\n=@set-switch{$1;1}
  ARGV:\N-*\n=@err{Unrecognized option\:\ "-*"\n}@exit-status{3}
\end{verbatim}

For an example of extending the command line options, suppose you wanted
to emulate a C pre-processor by accepting ``\verb/-D/'' options to
define macros.  That could be done by defining rules such as:
\begin{verbatim}
  ARGV:\N-D<I>\=*\n=@define{\\I$1\\I\=@quote{$2}}
  ARGV:\N-D<I>\n=@define{\\I$1\\I\=1}
\end{verbatim}

Instead of adding to the built-in rules, it is also possible to suppress
the built-in rules and define your own rules from scratch.  To do this,
start the program with a command line like:\newline
\verb|  gema -prim |{\em pattern-file  \ldots}
\newline
The \verb/-prim/ (``primitive mode'') option suppresses loading of the
built-in rules and reads patterns from the specified file.  Then the
remainder of the command line is processed according to whatever {\tt
ARGV} rules were defined in that file.
Note that even the default behavior of reading from standard input and
writing to standard output is implemented by the {\tt ARGV} rules.
(The \verb/-prim/ option is the only one that is hard-coded instead of
being implemented by patterns.)

\section{Status and Future development}

This program is now at the stage where in an ideal world it should be
regarded as a completed prototype, and it would now be the time to start
designing the real program to replace it.  However, as usually happens
in the real world, we ship the prototype because there isn't time to do
any more.  There is room for improvement in the areas of consistency,
ease of use, and performance at least.
Also, this documentation was written rather hurriedly and is not nearly as
polished as I would like.

Since this was developed by one person as a spare time hobby, it has not
had very extensive testing, so there are likely to be bugs.  The
\verb/-w/ and \verb/-t/ options are the most recently added
functionality, and hence the most likely to have inadequacies.

I don't know at this time whether I will be spending any more effort on
further development, but I am interested in hearing about any bugs found
or other suggestions.

Following, in no particular order, are some assorted ideas for
enhancements which remain for the future:
\begin{itemize}
\item Should warn about a domain that is defined but not referenced,
since it it easy to mistakenly neglect to quote a colon.
\item It might be useful to have a way to switch 
(or {\em push} and {\em pop}) the output
file -- e.g. to write each chapter of a document to a separate file
even though the input might be a single file.
\item A function to construct a unique pathname for a temporary file.
\item Separate pre-defined recognizers for lower case and upper case letters.
\item A notation for quoting a long section of
literal text, in addition to using the backslash for quoting individual
characters.
\item A function to return the pathname fo the current directory.
\item Record the file and line that each rule came from, to be used in
run-time error messages.
\item Provide a trace mode to list what
templates are and are not matched as an aid for debugging pattern files.
\item A template operator for specifying an action to be taken after all
input files have been processed.
\item A ``sleep'' function.
\end{itemize}

\section{Acknowledgments}

This program was conceived as an extension of the concepts
embodied in W. M. Waite's ``STAGE2'' processor\footnote{{\em
Communications of the ACM}, July, 1970, page 415ff}, as implemented by
Roger Hall.\footnote{Under the name ``TILT'' (``Texas Instruments
Language Translator''), it was used on various TI computers during the
70s and 80s.}

This program has some similarities to {\tt awk}, but
they are generally due more to similarity of purpose than to any
deliberate copying.  I did copy the \verb/$0/ notation and adopt the term
{\em action}.

This program was designed and coded by myself, David N. Gray, except for the
regular expression processor, which utilizes public domain code written
by Ozan S. Yigit and updated by Craig Durland and Harlan Sexton.
