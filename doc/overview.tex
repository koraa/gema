
% $Id$

\section{Operational Overview}

The general model of operation is that the program reads an input file
and writes an output file which consists of the input data transformed in
accordance with a set of transformation rules provided by the user.
A rule consists of a {\em template} and an {\em action}.  The template
is a pattern which the program will attempt to match with the input data.
Any input text that matches a template pattern will be replaced by the
result of evaluating the rule's action.  There may be multiple sets of
rules, where each set of rules is called a {\em domain}.  At any given
time, translation is controlled by the rules of one particular domain, but
both templates and actions are able to switch to a different domain for
processing particular portions of the data.  A domain can inherit from
another domain, meaning that if no match is found for the current input
text in any of the rules for the current domain, then the rules of the
inherited domain will be tried.

Processing of a file begins using the default domain, whose name is the
empty string.  First, if there is a rule with template ``\verb/\B/''
(beginning of file) or ``\verb/\A/'' (beginning of data), then its action
is performed.  Then the program begins reading the file.  For each
character position in the file, the program attempts to find a rule in
the current domain whose template pattern matches the input text
beginning at that point.  If a match is found, then the input stream is
advanced to the end of the matched text and the rule's action is
executed.  When no template matches the current position, the current
character is copied to the output file (unless the \verb/-match/ option
is being used), the input stream is advanced to the next character, and
it tries to find a template matching the text starting at that position.
When the end of the input file is reached, if there are any rules with
template ``\verb/\E/'' (end of file) or ``\verb/\Z/'' (end of data),
their actions will be executed, and then the files will be closed.

However, if a template matches without advancing the input stream (for
example, if it begins with ``\verb/\P/''), then after executing its action,
the search continues as though it had not matched.  This is necessary to
avoid hanging in a loop repeating the same match forever.

A rule may have an empty action, with the effect that the matching text is
simply discarded.  In each domain there may be at most one rule with an
empty template, which signifies a default action to be taken when no other
rule matches.  However, since an empty template does not cause the input
stream to be advanced and there are no more rules to try, this is only
meaningful if the corresponding action exits the current context by using
one of \verb/@end/, \verb/@terminate/, \verb/@fail/, or \verb/@abort/.

Generally speaking, while looking for a match, the rules within a domain will
conceptually be tried in the same order in which they were defined, so
wherever there might be ambiguities, the user should define the rules
for preferred special cases before the rules for default general cases.
However, there are some important exceptions:
\begin{itemize}
\item Rules beginning with a literal character (including space or
``\verb/\S/'') will be tried before rules beginning with an argument.
(Operators that don't advance the input stream, such as ``\verb/\N/'',
``\verb/\I/'', or ``\verb/\L/'', are ignored for the purpose of this
determination.)  The way this actually works is that the current input
character is used as an array index to find the list of rules beginning
with that character.  If none are found, or none match, then the rules
beginning with arguments are tried in sequence.

\item If two rules begin with the same sequence of literal characters,
the one with the longer literal string will be tried first.  This is
because if this were not done, a more specific rule appearing later
would never match, which would surely not be what the user intended.

\item If two rules have identical templates, the second rule will replace the
first on the assumption that it is a redefinition.\footnote{This policy
was adopted before {\tt @undef} had been invented; it might be better to
warn about duplicates and require explicit undefinition before redefining.}
\end{itemize}

Rules can be defined either as arguments of the \verb/-p/ command line
option or in pattern files loaded by the \verb/-f/ option.  Each line in
a pattern file can be one of the following:
\begin{itemize}
\item A comment line, indicated by a ``\verb/!/'' in the first column.
\item Any blank lines are ignored.
\item An immediate action -- a line beginning with ``\verb/@/'' can
contain one or more function calls which will be evaluated immediately
before reading the next line.  Normally these should be functions that
are used for their side effects and do not return a value.  The most
common case is using ``\verb/@set/'' to initialize a variable.
\item Any other line is expected to be a rule, which should contain a
``\verb/=/'' separating the template and action.  A line may contain
multiple rules, separated by semicolons.  A rule may be continued on
another line by ending the line with a backslash, which causes the
following newline to be ignored and skips over any leading spaces on the
following line.

Rules may be preceded by a domain name followed by a colon, which causes
all of the rules following on the same line to be defined in the
designated domain.  The domain name may optionally be enclosed in angle
brackets, and any leading or trailing blanks will be ignored.
If the domain is to be referenced in an action following ``\verb/@/'', then
the name should be limited to letters, digits, hyphen, and underscore.
\item A line of the form ``{\em name1}{\tt ::}{\em name2}'' specifies
that the domain named on the left inherits from the domain named on the
right.  Each domain can inherit from at most one other, but multiple levels
of inheritance is allowed.
\end{itemize}

A template may contain any of the following:
\begin{itemize}
\item Literal characters, which are to be exactly matched.
Literal characters include letters, digits, special characters quoted by
a backslash, control characters designated by backslash followed by
a lower case letter or digit, control characters designated by
``\verb/^/'' followed by a letter, and any other characters that don't have
any special meaning.  Any of the 256 possible characters can be used.
\item Arguments, which match some variable portion of text, and remember
the matched text so that it can be referenced later.
There are several different kinds of arguments, which are denoted by
``\verb/*/'', ``\verb/?/'', ``\verb/#/'', ``{\tt <}{\em \ldots}{\tt >}''.
and ``{\tt /}{\em \ldots}{\tt /}''.
The current implementation allows a template to have a maximum of twenty
arguments.
\item Template operators, denoted by a backslash followed by an upper
case letter, or by the space character.  These may set local options,
impose additional
requirements for a match, or allow skipping of redundant spaces in
the input.
\item A dollar sign may be used followed by a single letter or digit to
insert the value of a variable or a previous argument into the text that
is to be matched.  (This does not currently work for ``\verb/*/'' arguments.)
\end{itemize}

An action may contain any of the following:
\begin{itemize}
\item Literal characters, which when evaluated simply copy themselves to
the output stream.
\item The value of a template argument can be output by using a dollar
sign followed by the argument number (enclosed in braces if more than
one digit), or by using the characters ``\verb/*/'', ``\verb/?/'',
or ``\verb/#/'' to denote the corresponding argument.
\item The notation ``\verb/${/{\em name}\verb/}/'' can be used to output
the value of a variable.
\item The notation ``\verb/@/{\em name}\verb/{/{\em args}\verb/}/'' can
be used to call a built-in function or to translate the argument with a
user-defined domain.
\item Some of the operators denoted by a space or backslash followed by an upper
case letter also apply in actions, typically for conditional output of
spaces or newlines.
\end{itemize}
